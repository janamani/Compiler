\chapter{Conclusion and Future Work}


\section{Conclusion}

\par
\hspace{1.2cm} In this report, a 8-bit A to D converter is proposed , which is designed in CMOS UMC 180 nanometer technology, with semi-custom design by using Farady design kit. The proposed successive approximation A to D converter can reduce power by reducing the activity of the circuit. It is a hybrid of both level cross sampling and Nyquist sampling. The sampling takes place in track and hold circuit as per Nyquist sampling but the conversion process is initiated by level cross sampling method. Further number of conversion cycles in proposed successive approximation A to D converter can be reduced by adjusting the preset data to the comparator array.


\section{Future Work}

\par
\hspace{1.2cm} Future work includes completion of proposed successive A to D converter and send it for fabrication and testing of it. The activity of the circuit depends on the resolution of analog to digital converter, so a variable resolution analog to digital converter will reduce more power, when the low resolution is sufficient for the application, to implement this the difference quantificator has the option to change it's window size by external input so that the conversion process initiated by the difference quantificator will be reduced. 



